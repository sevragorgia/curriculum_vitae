\documentclass[letter,10pt]{article}
\usepackage[T1]{fontenc}
\usepackage{pxfonts}
\usepackage[pdftex]{color,graphicx}
\usepackage{authblk}
\usepackage[pdftex,colorlinks,urlcolor=blue]{hyperref}
\usepackage{eso-pic}
\usepackage[pdftex]{geometry}
\usepackage{textcomp}

\hyphenation{}
\pagestyle{headings}
\setlength{\parindent}{0pt}

\newcommand\monograma{%coloca un monograma en el centro de la pagina
	\put(0,0){
		\parbox[b][\paperheight]{\paperwidth}{
			\vfill
			\centering
			\includegraphics[width=\paperwidth, height=\paperheight, keepaspectratio]{../../monograma/monograma.png}
			\vfill			
		}
	}
}

\begin{document}
\AddToShipoutPicture{\monograma}

\subsection*{\textsf{Sergio Vargas Ram\'irez}}

Birth date: November 17th, 1980.\\
Birth place: San Jos\'e, Costa Rica.\\
Current Address: Nadistr. 6, D-80809, Munich, Germany\\
%Permanent Address: Apdo. 26-1100, Tib\'as, San Jose, Costa Rica.\\
Tel: +49 (0)163 4677860\\
%Tel: +506 8536 4900\\
Civil status: Married \\
Children: 2 \\
%\url{www.marinemolecularevolution.org}\\
%\href{s.vargas@marinemolecularevolution.org}{s.vargas@marinemolecularevolution.org}\\
\href{mailto: s.vargas@lrz.uni-muenchen.de}{s.vargas@lrz.uni-muenchen.de}\\
\href{mailto: sergio.vargas@lmu.de}{sergio.vargas@lmu.de}
\newline
\href{https://scholar.google.de/citations?user=OAcpy4EAAAAJ&hl=de&oi=sra}{Google Scholar Profile}\\
\href{http://www.researcherid.com/rid/A-5678-2011}{ResearcherID}\\


\begin{abstract}
{\noindent}Experience on systematics, biogeography and evolution of marine organisms, especially sponges and corals. Knowledge on molecular biology and protein chemistry techniques, e.g. PCR, DNA cloning, and Sanger and NG sequencing, recombinant protein expression and purification. Proficiency in univariate and multivariate statistics and experimental design. Bioinformatics, and intermediate-advance level computer programming (e.g. Python, Perl, R) and database management experience. Experience in project design and management.\\
\end{abstract}

\subsubsection*{Current position}
\begin{description}
\item [2013--present] Assistant Professor (Wiss.~Assistent) at the Chair of Paleontology \& Geobiology, Dept.~of Geo- and Environmental Sciences, Geobiology and Palaeontology, LMU M\"unchen, Germany
\end{description}

\subsubsection*{Formal Education}
\begin{description}
\item[1998--2002] Bachellor Degree in Biology. Universidad de Costa Rica.

\item[2003--2007] Magister Scientiae Degree in Biology. Universidad de Costa Rica. Honors Thesis.

\item[2009--2012] Promotion (Dr. rer. nat.). Dept.~of Geo- and Environmental Sciences, Geobiology and Palaeontology, LMU M\"unchen, Germany. \emph{Summa cum laude}.
\end{description}

\subsubsection*{Working Experience}
\begin{description}
\item [2012--2013] Service Engineer, IT Services, Perkin-Elmer Inc.\\Software specialist (expert user) for Perkin-Elmer's ELN BioAssay. Consultant on laboratory automation and e-management.

\item[2009--2012] Research Associate. Molecular Geo- and Palaeobiology Lab., Dept.~of Geo- and Environmental Sciences, Geobiology and Palaeontolgy, LMU M\"unchen, Germany.\\ Evolution of Antarctic sponges: molecular phylogeny and DNA barcoding of Antarctic sponges.
Head researcher, Prof. Gert W\"orheide, LMU M\"unchen.

\item[Sept. 2006--Jan. 2007] Short-term Fellow. Molecular phylogenetics of the eastern Pacific genus \emph{Heterogorgia}, Verrill and related taxa. ITZ, Tier\"arzliche Hochschule Hannover. Hannover, Germany.\\
ITZ director: Prof. Dr. Bernd Schierwater, TiHo, Hannover.

\item[2004--2008] Research Assistant. Systematics and biogeography of eastern Pacific octocorals\\
Head researcher: Hector M. Guzman Ph.D. Smithsonian Tropical Research Institute, Panam\'a.

\item[2004] Scientific Staff: Lab.~director assistant.~La Selva Biological Station, OTS.~Lab.~administrative maintenance, academic support for undergraduate and graduate students and courses.\\
Lab. director, La Selva Biological Station: Mahmood Sasa Ph.D. ICP, UCR.

%\item[2002--2003] Research Assistant.~Instituto Clodomiro Picado, UCR. Molecular biology and evolution of snake venom PLA2 toxins.\\
%Head researcher: Dr. Alberto Alape Ph.D. ICP, UCR.

%\item[2001--2002] Research Assistant. Rice biotechnology group, Centro de Investigaciones en Biolog\'ia Celular y Molecular, UCR.  Statistical analysis and database maintenance.\\
%Head researcher: Dr. Ana Mercedes Espinoza Ph.D.  CIBCM, UCR.

%\item[1999--2000] Research Assistant.~HERMA Project, Centro de Investigaciones Marinas y Limnol\'ogicas, UCR. Lab. assistant, CARICOMP project. Field and lab. assistant, Bah\'ia Culebra monitoring program.\\
%Head researcher: Dr. Jorge Cort\'es Ph.D. CIMAR, UCR.
\end{description}

%\subsubsection*{Teaching Experience}
%\begin{description}
%\item[2013--present] Assistant Professor. Courses: Geostatistics (Introductory statistics for Geo- and Paleobiologists; Master), Geomicrobiology (Laboratory; Master), Evolution (Bachelor and Master), Data analysis for Geo- and Paleobiologists (Bachelor)

%\item[2005--2006] Teaching Assistant. Course: Statistics. Graduate School in Microbiology, Universidad de Costa Rica. Professor in charge: Dr. Mahmood Sasa Ph.D., ICP, UCR.

%\item[2004] Teaching Assistant. Course: Undergraduate Semester Abroad Program (USAP). Organization for Tropical Studies/Duke University. Professor in charge: Dr. Erika Deinert, OTS; Dr. Mahmood Sasa Ph.D., ICP, UCR.

%\item[2003] Teaching Assistant. Course: Molecular Techniques in Tropical Ecology. Organization for Tropical Studies. Professor in charge: Dr. Mahmood Sasa Ph.D., ICP, UCR.
%\end{description}

\subsubsection*{Peer reviewed published articles\footnote{M.Sc.=Results from my Masters thesis; Dr.rer.nat.=Results from my doctoral dissertation; *=Corresponding author}\textsuperscript{,} \footnote{Underlined=undergraduate students during the study}}%\footnote{Peer reviewed abstracts and talks can be found at \url{http://www.marinemolecularevolution.org/publications}}}

\begin{description}
\item[]\textbf{Sergio Vargas}, Michelle Kelly, Kareen Schnabel, Sadie Mills, David Bowden \& Gert W\"orheide. Diversity in cold hot-spot: DNA-barcoding reveal patterns of evolution among Antarctic sponges (Phylum Porifera). \textbf{\emph{PLoS ONE}}, 10: e0127573\hfill\textbf{{\scriptsize (16) Dr.rer.nat.}}

\item[]Monika Bryce, Angelo Poliseno, Philip Alderslade, \textbf{Sergio Vargas}. Digitate and capitate soft corals (Cnidaria: Octocorallia: Alcyoniidae) from Western Australia with reports on new species and new Australian geographical records. 2015. \textbf{\emph{Zootaxa}} 3963: 160--200\hfill\textbf{{\scriptsize (15)}}

\item[]\textbf{Sergio Vargas*}, Hector M. Guzm\'an, Odalisca Breedy \& Gert W\"orheide. Molecular phylogeny and DNA-barcoding of tropical eastern Pacific gorgonian octocorals (Octocorallia: Gorgonacea). 2014. \textbf{\emph{Marine Biology}}, 161: 1027--1038\hfill\textbf{{\scriptsize (14)}}

\item[]Amir Szitenberg, Leontine E. Becking, \textbf{Sergio Vargas}, Julio C.~C. Fernandez, Nadiezhda Santodomingo, Gert W\"orheide, Micha Ilana, Michelle Kelly \& Doroth\'ee Huchon. Phylogeny of Tetillidae (Porifera, Demospongiae, Spirophorida) based on three molecular markers. 2013. \textbf{\emph{Molecular Phylogenetics and Evolution}}, 67: 509--519\hfill\textbf{{\scriptsize (13)}}

\item[]Martin Dohrmann, \textbf{Sergio Vargas}, Dorte Janussen, Allen G. Collins \& Gert W\"orheide. 2013. Molecular paleobiology of early-branching animals: integrating DNA and fossils elucidates the evolutionary history of hexactinellid sponges. \textbf{\emph{Palaeobiology}}, 39: 95--108. \hfill\textbf{{\scriptsize (12)}}

\item[]\textbf{Sergio Vargas}, Dirk Erpenbeck, Christian G\"ocke, Kathryn Hall, John N. A. Hooper, Dorte Janussen \& Gert W\"orheide. 2013. Molecular phylogeny of \emph{Abyssocladia} (Cladorhizidae: Poecilosclerida) and \emph{Phelloderma} (Phellodermidae: Poecilosclerida) suggest a diversification of chelae micro-spicules in cladorhizid sponges. \textbf{\emph{Zoologica Scripta}}, 42: 106--116. \hfill\textbf{{\scriptsize (11) Dr.rer.nat.}}

\item[]Odalisca Breedy, Leen van Ofwegen \& \textbf{Sergio Vargas}. 2012. A new family of soft corals (Anthozoa, Octocorallia, Alcyonacea) from the aphotic tropical eastern Pacific waters revealed by integrative taxonomy. \textbf{\emph{Systematics \& Biodiversity}}, 10: 351--359. \hfill\textbf{{\scriptsize (10)}}

\item[]\textbf{Sergio Vargas}, \underline{Astrid Schuster}, \underline{Katharina Sacher}, Gabrielle B\"uttner, Simone Sch\"atzle, Benjamin L\"auchli, Dirk Erpenbeck \& Gert W\"orheide. 2012. Barcoding sponges: an overview based on comprehensive sampling. \textbf{\emph{PLoS ONE}}, 7: e39345. \hfill\textbf{{\scriptsize (9) Dr.rer.nat.}}

%\item[]Dirk Erpenbeck, Kathryn Hall, Gabriele B\"uttner, \underline{Katharina Sacher}, Simone Sch\"atzle, \underline{Astrid Schuster}, \textbf{Sergio Vargas},  John N.A. Hooper \& Gert W\"orheide. 2012. The phylogeny of halichondrid demosponges: Past and present re-visited with DNA-Barcoding data. \textbf{\emph{Organisms Diversity and Evolution}}, 12: 57--70. \hfill\textbf{{\scriptsize (8)}}

%\item[]Gert W\"orheide, \textbf{Sergio Vargas}, Carsten L\"uter \& Joachim Reitner. 2011. Precious corals and sponge rock gardens on the deep aphotic fore-reef of Osprey Reef (Coral Sea, Australia). \textbf{\emph{Coral Reefs}}, 30: 901. \hfill\textbf{{\scriptsize (7)}}

\item[]\textbf{Sergio Vargas*}, Michael Eitel, Odalisca Breedy \& Bernd Schierwater. 2010. Molecules match morphology: mitochondrial DNA supports Bayer's \emph{Lytreia-Bebryce-Heterogorgia} (Alcyonacea: Octocorallia) clade hypothesis. \textbf{\emph{Invertebrate Systematics}}, 24: 23--31. \hfill\textbf{{\scriptsize (6)}}

%\item[]\textbf{Sergio Vargas*}, Odalisca Breedy \& Hector M. Guzman. 2010. The phylogeny of \emph{Pacifigorgia} (Coelenterata, Octocorallia, Gorgoniidae): A case of study of the use of continuous characters in the systematics of the Octocorallia. \textbf{\emph{Zoosystema}}, 32: 5--18. \hfill\textbf{{\scriptsize (5) M.Sc.}}

%\item[]\textbf{Sergio Vargas*}, Odalisca Breedy, Francisco Siles \& Hector M. Guzman. 2010. How many kinds of sclerite?: Towards a morphometric classification of gorgoniid microskeletal components. \textbf{\emph{Micron}}, 41: 158--164. \hfill\textbf{{\scriptsize (4) M.Sc.~On the cover!}}

%\item[]Odalisca Breedy, Hector M. Guzman \& \textbf{Sergio Vargas}. 2009. A revision of the genus \emph{Eugorgia} Verrill, 1868 (Coelenterata: Octocorallia: Gorgoniidae). \textbf{\emph{Zootaxa}}, 2151: 1--46. \hfill\textbf{{\scriptsize (3)}}

\item[]\textbf{Sergio Vargas*}, Hector M. Guzman \& Odalisca Breedy. 2008. Distribution patterns of the genus \emph{Pacifigorgia} (Octocorallia, Gorgoniidae): Track analysis and parsimony analysis of endemicity. \textbf{\emph{Journal of Biogeography}}, 35: 241--247. \hfill\textbf{{\scriptsize (2) M.Sc.}}

%\item[]Griselda Arrieta--Espinoza, Elena Sanchez, \textbf{Sergio Vargas}, Jorge Lobo, Tania Quesada \& Ana M. Espinoza. 2005. The Weedy Rice Complex in Costa Rica. I. Morphological Study of Relationships Between Commercial Rice Varieties, Wild Oryza Relatives and Weedy Types. \textbf{\emph{Genetic Resources and Crop Evolution}}, 52: 575--587. \hfill\textbf{{\scriptsize (1)}}
\end{description}

%\subsubsection*{Articles in press}
%\begin{description}
%\end{description}

%\subsubsection*{Accepted manuscripts}
%\begin{description}
%\end{description}

\subsubsection*{Submitted manuscripts}%\footnote{Manuscripts in preparation can be found at \url{http://www.marinemolecularevolution.org/publications}}}
\begin{description}
\item[]\underline{Sandra L. Ament}, Odalisca Breedy, Jorge Cort\'es, Hector M Guzman, Gert W\"orheide \& \textbf{Sergio Vargas*}. Homoplasious colony morphology and mito-nuclear phylogenetic discordance among Eastern Pacific octocorals. \textbf{\emph{Molecular Phylogenetics and Evolution}}

\item[]\underline{Sergio Tusso}, Kerstin Morcinek, Catherine Vogler, Peter J. Schupp,  Ciemon F. Caballes, \textbf{Sergio Vargas} \& Gert W\"orheide. Putative long distance gene flow and evidence of secondary outbreaks of the Crown-of-Thorns Starfish in the Pacific Ocean \textbf{\emph{PeerJ}}, \hfill\textbf{{\scriptsize \href{https://peerj.com/preprints/1167v1/}{Preprint}}}

%\item[]\textbf{Sergio Vargas}, Martin Dohrmann, Christian G\"ocke, Dorte Janussen \& Gert W\"orheide. Molecular phylogeny of the sponge genus \emph{Rossella} (Hexactinellida: Lyssacinosida, Rossellidae): one species and a species complex. \textbf{\emph{Polar Biology}} \hfill\textbf{{\scriptsize Dr.rer.nat.}}
\end{description}

\subsubsection*{Manuscripts in preparation}
\begin{description}

\item[]\textbf{Sergio Vargas}, Christian G\"ocke, Dorte Janussen \& Gert W\"orheide. Counting in the abyss: sponge species richness estimation in the deep Weddel Sea (Antarctica). \hfill\textbf{{\scriptsize Dr.rer.nat.}}

\item[]Angelo Poliseno, Gert W\"orheide \& \textbf{Sergio Vargas*}. Historical biogeography and mitogenomics of two endemic Mediterranean gorgonians (Plexauridae: Paramuricea).

\end{description}

%\subsubsection*{Peer-reviewed conference abstracts and talks}% \footnote{Abstracts and talks can be found at \url{http://www.marinemolecularevolution.org/publications}}}
%\begin{description}
%\item[]\textbf{Sergio Vargas}, Kareen Schnabel, Michelle Kelly, David Bowden \& Gert W\"orheide. 2011. Barcoding the Ross shelf sponge assemblages. IV International Barcode of Life Conference Adelaide, Australia.

%\item[]\textbf{Sergio Vargas} \& Gert W\"orheide. 2011. Molecular phylogeny of the carnivorous sponge family Cladorhizidae: implications for the systematics of Poecilosclerida. Deep Metazoan Phylogeny, M\"unchen, Germany.

%\item[]\textbf{Sergio Vargas}, Christian G\"ocke, Dorte Janussen \& Gert W\"orheide. 2010. A phylogenetic analysis of the genus \emph{Rossella} (Hexactinellida, Rossellidae). VII World Sponge Conference, Girona, Spain.

%\item[]\textbf{Sergio Vargas}, Dorte Janussen \& W\"orheide. 2010. Phylogenetic diversity of Antarctic sponges: Insights and prospects. VII World Sponge Conference, Girona, Spain

%\item[] \textbf{Sergio Vargas}, Dirk Erpenbeck, Astrid Schuster, Katharina Sacher, Gabrielle B\"uttner, Simone Sch\"atzle \& Gert W\"orheide. 2010. A high-throughput, low-cost Porifera DNA barcoding pipeline. VII World Sponge Conference, Girona, Spain

%\item[] \textbf{Sergio Vargas}, Dirk Erpenbeck, Dorte Janussen \& Gert W\"orheide. 2010. Palaeo-trans-Antarctic connections in \emph{'Polymastia invaginata'}? A case study using ribosomal (28S) DNA markers. Annual Meeting of the (German) Palaeontological Society, Munich, Germany.

%\end{description}

%\subsubsection*{Scholarships and grants}
%\begin{description}
%\item[2013] LMU Excellent Junior Research Fund: \emph{Pacifigorgia} Speciation Genomics: sympatry, modularity and speciation. Approx. award: \texteuro 12000.

%\item[2010] Boehringer Ingelheim Travel Fonds. EMBO Practical Course: Computational Molecular Evolution, Heraklion, Greece. Approx. award: \texteuro 600.

%\item[2009]  STIBET scholarship for international Ph.D. students. DAAD. Approx. award: \texteuro 1200. 

%\item[2007] Carolina Foundation scholarship for short-term summer courses. Course (in spanish): Bioinformatics and Computational Biology, Escuela Complutense de Verano, Universidad Complutense de Madrid. Madrid, Espa\~na. Approx. award: \texteuro 3000.

%\item[2006--2007] Short-term scholarship for young researchers and PhD students. DAAD. Student fellowship under the supervision of Prof.~Dr.~Bernd Schierwater, ITZ, Tier\"arzliche Hochschule Hannover. Hannover, Germany. Approx. award: \texteuro 4500.

%\item[2005] Research grant for master students. Sistema de Estudios de Posgrado, Universidad de Costa Rica. Approx. award: \$750.
%\end{description}

\subsubsection*{Miscellaneous information}
\begin{small}
	\begin{description}
		\item[Operating Systems \& Programming languages] GNU-Linux/Unix, Perl, Python, SQL, R, C/C++ (basic).
		\item[Reviewer for]
	\end{description}

	\begin{itemize}
		\item Molecular Phylogenetics and Evolution, Invertebrate Systematics, Oikos, Zoological Journal of the Linnean Society, BMC Evolutionary Biology, PLoS ONE, Proceedings of the Royal Society B.
	\end{itemize}

	\begin{description}
		\item[Languages]
	\end{description}

	\begin{itemize}
		\item English (TOEFL: 643; TWE: 4)
		\item German [ZD (B1): 247 (max. 300)].
	\end{itemize}

	%\begin{description}
	%	\item[References]
	%\end{description}
	%\begin{itemize}
	%	\item{Prof. Dr. Gert W\"orheide\newline Department of Earth- \& Environmental Sciences, Palaeontology and Geobiology\\ Ludwig-Maximilians-Universit\"at M\"unchen\newline GeoBio-Center$^{LMU}$\newline and Bavarian State Collections of Palaeontology and Geology\newline tel: +49 (0) 89 2180-6718\newline fax: +49 (0) 89 2180 6601\newline woerheide@lmu.de}
	%	\item{Hector M. Guzman, PhD.\\Smithsonian Tropical Research Institute\\tel: +507 212-8733\\fax:+507 212-8190/8791\\guzmanh@si.edu}

	%	\item{P.D. Dr. Dorte Janussen\newline Forschungsinstitut und Naturmuseum Senckenberg,\\ Senckenberganlage 25,\\ D-60325 Frankfurt am Main,\\Germany.\newline tel: +49 (0) 69 7542 1306\newline Dorte.Janussen@senckenberg.de}
	%\end{itemize}

	\begin{flushright}
		last reviewed: \today
	\end{flushright}
\end{small}

\begin{center}
	\vspace{\stretch{1}}
	{\scriptsize \texttt{typesetted using}}\\{\small \texttt{\LaTeXe}}\pagebreak
\end{center}
\end{document}

